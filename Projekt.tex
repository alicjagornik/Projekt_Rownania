\documentclass{article}
\usepackage{graphicx} % Required for inserting images
\usepackage{polski}
\usepackage[utf8]{inputenc}
\usepackage{amsmath}

\newcounter{mojlicznik}  % Tworzymy nowy licznik

\title{Równania różniczkowe - projekt}
\author{Jagoda Płócienniczak, Alicja Górnik}

\begin{document}
\setcounter{mojlicznik}{1}  % Ustawiamy wartość licznika na 1
\maketitle
\section*{PODSTAWOWA PRZEMIANA MATERII (BMR) - RÓWNANIE MIFFLINA ST JEORA}

\begin{enumerate}
    \item Ilość energii, którą organizm zużywa w spoczynku do podstawowych funkcji życiowych (np. oddychanie, regulacja temperatury ciała);
    \item równanie Mifflina jest uznane za jedno z najdokładniejszych do obliczenia BMR (Bascal metabolic rate): \\
    $ P = \left( \frac{10,0 m}{1 kg} + \frac{6,25 h}{1 cm} - \frac{5,0 a}{1 year} + s \right) \frac{kcal}{day}$, gdzie $m$ - masa [m], $h$ - wzrost [cm], $a$ - wiek [lat], $s = +5$ - dla mężczyzny i $s = -161$ - dla kobiety.
    \item Tempo zmiany masy ciała nie jest stałe, ponieważ wraz z utratą wagi zmienia się nasze BMR (zatem musimy dostosować ilość spożytych kalorii).\\
    Zatem skorzystamy z \Roman{mojlicznik} zasady termodynamiki:
	 \begin{align*}
    \Delta U = Q - W \text{, gdzie } &\Delta U \text{ - zmiana energii wewnętrznej systemu}\\
    	&Q\text{ - ciepło dostarczone do sytemu}\\
    	&W\text{ - praca wykonana przez system.}
	\end{align*}

    Wykorzystując \Roman{mojlicznik} zasadę termodynamiki w spalaniu kalorii:
    \begin{align*}
         \frac{\text{nadwyżka kalorii}}{\text{deficyt kalorii}} &= \text{kalorie spożyte} - \text{kalorie spalone} \text{, co jest równoważne:}\\
        \text{zmiana masy} &= \frac{n-T}{7700} = \frac{n - fP}{7700} = \frac{n - f\left( 10m\left(t\right) + 6,25h - 5a + s\right)}{7700}\\
        \frac{dm\left(t\right)}{dt} &= \frac{n - f\left( 10m\left(t\right) + 6,25h - 5a + s\right)}{7700} = \frac{fm\left( t \right)}{770} + \frac{n + f\left( 6,25h - 5a +s \right)}{7700}
    \end{align*}
    Niech $k = \frac{n + f\left( 6,25h - 5a +s \right)}{7700}$, ponieważ jest to pewna stała zależna od zadanych parametrów.
    \begin{align*}
        &\frac{dm\left(t\right)}{dt} = \frac{-fm\left(t\right)}{770} + k\\
        &m' + \frac{f}{770}m = k
    \end{align*}
    Niech $ \mu\left(t\right) = \mathrm{e}^{\left(\int \frac{f}{770} + k \right) \, dt\\} = \mathrm{e}^{\frac{f}{770}t}\  $ \\ 
    \begin{align*}
        m\mathrm{e}^{\frac{f}{770}t}\ &= k\int \mathrm{e}^{\frac{f}{770}t}\, dt\ \\
        m\mathrm{e}^{\frac{f}{770}t}\ &= k\frac{770}{f}
        \mathrm{e}^{\frac{f}{770}t}\ + C\\
        m &= k\frac{770}{f} + C \mathrm{e}^{-\frac{f}{770}t}\ = C\mathrm{e}^{-\frac{ft}{770}}\ + \left(\frac{n-f\left(6,25h - 5a + s\right)}{7700}\right)\frac{770}{f}\\
        m\left(0\right) &= C + \frac{n-f\left(6,25h - 5a + s\right)}{10f} \Rightarrow{C = m_0 - \frac{n-f\left(6,25h - 5a + s\right)}{10f}}\\
        m\left(t\right) &= \left( m_0 - \frac{n-f\left(6,25h - 5a + s\right)}{10f}\right)\mathrm{e}^{-\frac{ft}{770}}\ + \frac{n-f\left(6,25h - 5a + s\right)}{10f}
    \end{align*}


\end{enumerate}

\end{document}
